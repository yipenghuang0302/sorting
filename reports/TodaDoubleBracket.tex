% \documentclass{article}
% \usepackage[utf8]{inputenc}
% \usepackage{amsmath}

% \title{Toda Lattice, Double Bracket}
% \author{Lusa Zhan}
% \date{September 2016}

% \begin{document}

% \maketitle


% \section{Equivalence between ODE and Lie bracket notation}

The central idea of analog sorting as described in~\cite{bloch, brockett} makes use of the Toda lattice. The system of Hamiltonian equations in (1) associated with the Hamiltonian in (2) gives us the following set of equations
\begin{align*}
\dot{p_k} &= \exp(q_{k-1}-q_{k})-\exp(q_{k}-q_{k+1}) \\
\dot{q_k} &= p_k
\end{align*}
where the boundary conditions are set so $\exp(x_0-x_1)=\exp (x_n-x_{n+1})=0$.

This system is also analogous to the double bracket notation 
\begin{align}
\dot{H} = [H,[H,N]]
\end{align}

where the brackets stand for the Lie bracket $[A,B] = AB-BA$. The connection between the Hamiltonian system  and the double bracket notation of the Toda lattice will be discussed and outlined in the following sections.

\subsubsection{Toda lattice - ODE form}

The above system of ordinary differential equations can be transformed into the system described in~\cite{harvard_robo} using a change of variables.
\begin{align}
x_k &= -\frac{1}{2}p_k \nonumber \\
y_k &= \frac{1}{2}\exp(\frac{q_k-q_{k+1}}{2})
\end{align}

In that case, 
\begin{align*}
    \dot{x_k} &= -\frac{1}{2}\dot{p_k} \\
              &= -\frac{1}{2}(exp(q_{k-1}-q_{k})-exp(q_{k}-q_{k+1})) \\
              &= -\frac{1}{2}(4y^2_{k-1}-4y^2_k) \\
              &= 2y^2_k-2y^2_{k-1}
\end{align*}

and 
\begin{align*}
    \dot{y_k} &= \frac{1}{2}\exp(\frac{q_k-q_{k+1}}{2})\frac{\dot q_k-\dot q_{k+1}}{2} \\
              &= y_k\frac{p_k-p_{k+1}}{2} \\
              &= y_k(x_{k+1}-x_k) 
\end{align*}

Taking into account the boundary conditions $y_0=y_n=0$, we get the desired system of ODEs with
\begin{align}
    \dot{x_k} &= 2y^2_k-2y^2_{k-1} \nonumber \\
    \dot{y_k} &= y_k(x_{k+1}-x_k) \\
    y_0 &= y_n = 0 \nonumber
\end{align}

This is the system of ODEs we will solve in the analog chip and in our simulations.


\subsubsection{Toda lattice - Jacobi matrix}

The connection between the Toda lattice and the double bracket notation $\dot{H} = [H,[H,N]]$ can be made through the Jacobi Matrix form of the Toda lattice. The Jacobi matrix for the Hamiltonian system after the change of variables (4) is given by
\begin{align}
 H = \begin{bmatrix}
    x_{1} & y_{1} & 0  & \dots & 0 \\
    y_{1} & x_{2} & y_{2} & \dots & 0 \\
     & & \ddots & \\
          &       & y_{n-2} & x_{n-1} & y_{n-1}\\
    0 & \hdots & & y_{n-1} & x_{n}
\end{bmatrix}
\end{align}
This is the form of $H$ required for analog sorting as outlined by Brockett in ~\cite{brockett}.

In order to get the double bracket form, we need a diagonal matrix $N = \text{diag}(n, n-1, \dots, 1)$ whose role will be discussed in 2.5.1.

\begin{align*}
N = \begin{bmatrix}
        n & 0 & \hdots & 0 \\
        0 & n-1 & \\
        \vdots &  & \ddots & \vdots \\
        0 & \hdots & & 1
    \end{bmatrix}
\end{align*}

From this, we can calculate $[H[H,N]] = H[H,N]-[H,N]H$ step by step:

\[ 
HN = \begin{bmatrix}
        nx_1 & (n-1)y_1 & \hdots & 0 \\
        ny_1 & (n-1)x_2 & \hdots & 0 \\
        \vdots & & \ddots & 0\\
         &  & 2x_{n-1} & y_{n-1} \\
        0 & \hdots & 2y_{n-1} & x_n

    \end{bmatrix}
\]

\[ 
NH = \begin{bmatrix}
        nx_1 & ny_1 & \hdots & 0 \\
        (n-1)y_1 & (n-1)x_2 & \hdots & 0 \\
        \vdots & & \ddots & 0\\
         &  & 2x_{n-1} & 2y_{n-1} \\
        0 & \hdots & y_{n-1} & x_n

    \end{bmatrix}
\]

\[ 
HN-NH = \begin{bmatrix}
        0 & -y1 & \hdots &  & 0 \\
        y_1 & 0 & -y_2 & \hdots & 0 \\
        \vdots & \ddots & \ddots & \ddots & 0\\
         & & y_{n-2}& 0 & -y_{n-1} \\
        0 & & \hdots & y_{n-1} & 0

    \end{bmatrix}
\]

\[ 
H[H,N] = \begin{bmatrix}
        y_1^2 & -x_1y_1 & \hdots & 0 \\
        x_2y_1 & -y_1^2+y^2_2 &\hdots & 0 \\
        y_1y_2 & \ddots & & 0\\
        \vdots & & y^2_{n-1}-y^2_{n-2} & -x_{n-1}y_{n-1} \\
        0 & \hdots & x_ny_{n-1} & -y^2_{n-1}

    \end{bmatrix}
\]

\[ 
[H,N]H = \begin{bmatrix}
        y_1^2 & -x_2y_1 & \hdots & 0 \\
        x_1y_1 & y_1^2-y^2_2 &\hdots & 0 \\
        y_1y_2 & \ddots & & 0\\
        \vdots &  & -y^2_{n-1}+y^2_{n-2} & -x_{n}y_{n-1} \\
        0 & \hdots & x_{n-1}y_{n-1} & y^2_{n-1}

    \end{bmatrix}
\]
\\

Therefore, we get \\

\[[H,[H,N]] = 
\begin{bmatrix}
    2y_1^2 & y_1(x_2-x_1) & 0 \\
    y_1(x_2-x_1) & -2(y_1^2-y^2_2) & 0\\
    0 & \ddots & 0\\
    \vdots & & y_{n-1}(x_{n}-x_{n-1}) \\
    0 & \hdots & -2y^2_{n-1}
\end{bmatrix}
\]

This is equivalent to the result we get from combining the matrix form of $H$ with the values for $\dot{x}$ and $\dot{y}$
\[\dot{H} = \begin{bmatrix}
    \dot{x}_{1} & \dot{y}_{1} & 0  & \dots & 0 \\
    \dot{y}_{1} & \dot{x}_{2} & \dot{y}_{2} & \dots & 0 \\
     & & \ddots & \\
     & & \dot{y}_{n-2} & \dot{x}_{n-1} & \dot{y}_{n-1}\\
    0 & \hdots & & \dot{y}_{n-1} & \dot{x}_{n}
    
\end{bmatrix}\]

In total, we conclude that the double bracket notation of the Toda lattice used in other papers is analogous to the system of ODEs. We use the equivalent system of ODEs to construct the analog sorter.


