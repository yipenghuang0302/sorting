In this work, we validate that analog sorting works in practice.
While this idea has been presented several times in mathematical literature, this is the first where it is tested in practice, using analog computer hardware.
Using real measurements and simulated experiments, we evaluate the performance and efficiency of analog sorting.
We compare this technique against classical sorting algorithms.
In addition, we find limitations to analog sorting through experiments. Cases in which analog sorting could fail have not been discussed in previous literature.

In terms of concrete impact of our findings, analog sorting can be a new way to sort in future computer architectures.
Sorting is the underpinning of big data.
Increasingly, approximate sorts are useful.

In terms of the impact of this work on the future of analog computer techniques, we show in this paper two key ideas.
One, using the soliton behavior of the finite Toda lattice, we are able to swap two values in an analog computer without having to use a third capacitor as a buffer.
Two, it is possible to perform a typically ``discrete'' task using continuous-time hardware.
From this discrete primitive operation, we can build other discrete algorithms.


% What's exciting about this idea is it solves a deeply discrete problem using continuous methods.
% Furthermore, aside from this collection of references, the idea hasn't been explored elsewhere.