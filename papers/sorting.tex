While digital sorting algorithms are efficient, no prior work has discussed what the time complexity of analog sorting. To get a sense of how analog sorting performs, we need a basis for comparison. Some of the classical sorting algorithms are merge sort or quick sort. Those generally have nonlinear time complexity, but quick sort has logarithmic space complexity. Table 1 lists some algorithms and their respective complexities.

Later, we will analyze how analog sorting compares to digital sorting algorithms in terms of complexity. 

\begin{table}[h]
\centering
\caption{Complexities of Sorting Algorithms}
\begin{tabular}{|l|c|c|} \hline
Sorting Algorithm&Time Complexity&Space Complexity\\ \hline
Merge Sort & $O(n\log n)$& $O(n)$\\ \hline
Quick Sort & $O(n\log n)$& $O(\log n)$\\ \hline
Insertion Sort & $O(n^2)$& $O(1)$\\ \hline
Selection Sort & $O(n^2)$& $O(1)$\\
\hline\end{tabular}
\end{table}



% \begin{table*}
% \centering
% \caption{Some Typical Commands}
% \begin{tabular}{|c|c|l|} \hline
% Command&A Number&Comments\\ \hline
% \texttt{{\char'134}alignauthor} & 100& Author alignment\\ \hline
% \texttt{{\char'134}numberofauthors}& 200& Author enumeration\\ \hline
% \texttt{{\char'134}table}& 300 & For tables\\ \hline
% \texttt{{\char'134}table*}& 400& For wider tables\\ \hline\end{tabular}
% \end{table*}
% end the environment with {table*}, NOTE not {table}!