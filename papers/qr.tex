The analog sorting method we show is related to the classical QR algorithm.
While the QR algorithm is a discrete algorith operating step-by-step, the analog sorting algorithm does so in continuous time~\cite{deift, chu_realization, chu_flows}.

The QR algorithm finds the eigenvalues and eigenvectors of a square matrix.
The eigenvalue problem is as follows:
\[A_0x = \lambda x\]

The QR algorithm operates as follows:
Given the QR decomposition, each step proceeds as:
\[A_k = Q_k R_k\]
\[A_{k+1} = R_k Q_k\]

A side effect of the QR algorithm is that the eigenvalues of the original matrix end up in sorted order along the diagonal.
This is a useful property in eigenvalue problems, where users of the algorithm are interested in finding the largest few eigenvalues.
But few researchers have pointed out that this may be useful in itself, for sorting.
Other algorithms for finding the eigenvalues and eigenvectors include the Jacobi eigenvalue algorithm, and the divide-and-conquer algorithm.
It seems that the QR algorithm is the algorithm among these that sorts the eigenvalues.
