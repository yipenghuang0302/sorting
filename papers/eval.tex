How does analog sorting compare to other sorting algorithms for example quick, merge etc in terms of complexity?

\subsection{Functionality validation of analog sorting}
\begin{figure}
\centering
\includegraphics[width=\columnwidth]{../pyscripts/Graphs/10d_toda_2.png}
\caption{Simulated sorting a 10 element vector.}
\end{figure}

\subsection{Time cost of the discrete QR algorithm}
Time cost of overall QR loop:
	how many iterations of qr til convergence?
	% time to convergence of qr w.r.t.. problem size
	Since we know the ODE is analogous to QR algorithm, they should take the same amount of time.
Time cost of QR step:
	Numerical Recipes 3rd Edition p585 says the QR algorithm takes O(N) time for symmetric tridiagonal matrices.

\subsection{Time cost of analog sorting}

In terms of time, the sorter takes at least O(N) time because of the time it takes just for signals to propagate across the circuit.
Another issue is the time it takes for the ODE to settle to its final value.
This is preliminary data showing the time to convergence of analog sorting, plotted against problem size.

\begin{figure}
\centering
\includegraphics[width=\columnwidth]{../data/ode_time_vs_problem_size.pdf}
\caption{Preliminary time to convergence vs. problem size.}
\end{figure}

The axes are the time to solution, plotted against N, the number of real numbers we are sorting.
The data points come from multiple random trials at each N.
The set of real numbers is randomly generated from a chosen dynamic range.
It appears that as the N size increases, the average time grows linearly with respect to N.
Furthermore, the variance of the solution time, measured as the standard deviation, also grows.

Other variables to consider include:
\begin{enumerate}
\item the dynamic range of real numbers to sort
\item the initial values of the off-diagonals
\end{enumerate}

\subsection{Hardware cost of analog sorting}

The analog sorter takes up O(N) amount of circuit components to sort N elements.

% \begin{figure}
% \centering
% \includegraphics[height=1in, width=1in]{fly}
% \caption{A sample black and white graphic
% that has been resized with the \texttt{includegraphics} command.}
% \end{figure}

% \begin{figure*}
% \centering
% \includegraphics{flies}
% \caption{A sample black and white graphic
% that needs to span two columns of text.}
% \end{figure*}

% \begin{figure}
% \centering
% \includegraphics[height=1in, width=1in]{rosette}
% \caption{A sample black and white graphic that has
% been resized with the \texttt{includegraphics} command.}
% \vskip -6pt
% \end{figure}