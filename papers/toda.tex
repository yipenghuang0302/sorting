The finite Toda lattice is an ODE system that is related to the QR algorithm.
If you plot the evolution of the Toda lattice ODE with respect to time, the values of the ODE at integer time steps is the intermediate states of the QR algorithm.
% Bloch and Rojo: Section 3.2 of Bloch and Rojo paper says the Toda flow preserves the eigenvalues. This seems to be the main connection to the QR algorithm, which also preserves eigenvalues across iterations.
The common property of the finite Toda lattice and the QR algorithm is that both preserve the eigenvalues of the matrix they operate on~\cite{bloch}.

The finite Toda lattice is a Hamiltonian system with the form:
\begin{align}
\hham(p,q) = \frac{1}{2} \sum^{n}_{1}p_k^2 + \sum^{n-1}_{1}exp(q_k-q_{k+1})
\end{align}

This basic form can be changed in two ways, using change of variables, into the equations for analog sorting described in~\cite{brockett}.

% \documentclass{article}
% \usepackage[utf8]{inputenc}
% \usepackage{amsmath}

% \title{Toda Lattice, Double Bracket}
% \author{Lusa Zhan}
% \date{September 2016}

% \begin{document}

% \maketitle


% \section{Equivalence between ODE and Lie bracket notation}

The central idea of analog sorting as described in~\cite{bloch, brockett} makes use of the Toda lattice. The system of Hamiltonian equations in (1) associated with the Hamiltonian in (2) gives us the following set of equations
\begin{align*}
\dot{p_k} &= \exp(q_{k-1}-q_{k})-\exp(q_{k}-q_{k+1}) \\
\dot{q_k} &= p_k
\end{align*}
where the boundary conditions are set so that $\exp(x_0-x_1)=\exp (x_n-x_{n+1})=0$.

This system is also analogous to the double bracket notation 
\begin{align}
\dot{H} = [H,[H,N]]
\end{align}

where the brackets stand for the Lie bracket $[A,B] = AB-BA$. The connection between the Hamiltonian system  and the double bracket notation of the Toda lattice will be discussed and outlined in the following sections.

\subsubsection{Toda lattice - ODE form}

The above system of ordinary differential equations can be transformed into the system described in~\cite{harvard_robo} using a change of variables.
\begin{align}
x_k &= -\frac{1}{2}p_k \nonumber \\
y_k &= \frac{1}{2}\exp(\frac{q_k-q_{k+1}}{2})
\end{align}

In that case, 
\begin{align*}
    \dot{x_k} &= -\frac{1}{2}\dot{p_k} \\
              &= -\frac{1}{2}(exp(q_{k-1}-q_{k})-exp(q_{k}-q_{k+1})) \\
              &= -\frac{1}{2}(4y^2_{k-1}-4y^2_k) \\
              &= 2y^2_k-2y^2_{k-1}
\end{align*}

and 
\begin{align*}
    \dot{y_k} &= \frac{1}{2}\exp(\frac{q_k-q_{k+1}}{2})\frac{\dot q_k-\dot q_{k+1}}{2} \\
              &= y_k\frac{p_k-p_{k+1}}{2} \\
              &= y_k(x_{k+1}-x_k) 
\end{align*}

Taking into account the boundary conditions $y_0=y_n=0$, we get the desired system of ODEs with
\begin{align}
    \dot{x_k} &= 2y^2_k-2y^2_{k-1} \nonumber \\
    \dot{y_k} &= y_k(x_{k+1}-x_k) \\
    y_0 &= y_n = 0 \nonumber
\end{align}

This is the system of ODEs we will solve in the analog chip and in our simulations.


\subsubsection{Toda lattice - Jacobi matrix}

The connection between the Toda lattice and the double bracket notation $\dot{H} = [H,[H,N]]$ can be made through the Jacobi Matrix form of the Toda lattice. The Jacobi matrix for the Hamiltonian system after the change of variables (4) is given by
\begin{align}
 H = \begin{bmatrix}
    x_{1} & y_{1} & 0  & \dots & 0 \\
    y_{1} & x_{2} & y_{2} & \dots & 0 \\
     & & \ddots & \\
          &       & y_{n-2} & x_{n-1} & y_{n-1}\\
    0 & \hdots & & y_{n-1} & x_{n}
\end{bmatrix}
\end{align}
This is the form of $H$ required for analog sorting as outlined by Brockett in ~\cite{brockett}.

In order to get the double bracket form, we need a diagonal matrix $N = \text{diag}(n, n-1, \dots, 1)$ whose role will be discussed in 2.5.1.

\begin{align*}
N = \begin{bmatrix}
        n & 0 & \hdots & 0 \\
        0 & n-1 & \\
        \vdots &  & \ddots & \vdots \\
        0 & \hdots & & 1
    \end{bmatrix}
\end{align*}

From this, we can calculate $[H[H,N]] = H[H,N]-[H,N]H$ step by step:

\[ 
HN = \begin{bmatrix}
        nx_1 & (n-1)y_1 & \hdots & 0 \\
        ny_1 & (n-1)x_2 & \hdots & 0 \\
        \vdots & & \ddots & 0\\
         &  & 2x_{n-1} & y_{n-1} \\
        0 & \hdots & 2y_{n-1} & x_n

    \end{bmatrix}
\]

\[ 
NH = \begin{bmatrix}
        nx_1 & ny_1 & \hdots & 0 \\
        (n-1)y_1 & (n-1)x_2 & \hdots & 0 \\
        \vdots & & \ddots & 0\\
         &  & 2x_{n-1} & 2y_{n-1} \\
        0 & \hdots & y_{n-1} & x_n

    \end{bmatrix}
\]

\[ 
HN-NH = \begin{bmatrix}
        0 & -y1 & \hdots &  & 0 \\
        y_1 & 0 & -y_2 & \hdots & 0 \\
        \vdots & \ddots & \ddots & \ddots & 0\\
         & & y_{n-2}& 0 & -y_{n-1} \\
        0 & & \hdots & y_{n-1} & 0

    \end{bmatrix}
\]

\[ 
H[H,N] = \begin{bmatrix}
        y_1^2 & -x_1y_1 & \hdots & 0 \\
        x_2y_1 & -y_1^2+y^2_2 &\hdots & 0 \\
        y_1y_2 & \ddots & & 0\\
        \vdots & & y^2_{n-1}-y^2_{n-2} & -x_{n-1}y_{n-1} \\
        0 & \hdots & x_ny_{n-1} & -y^2_{n-1}

    \end{bmatrix}
\]

\[ 
[H,N]H = \begin{bmatrix}
        y_1^2 & -x_2y_1 & \hdots & 0 \\
        x_1y_1 & y_1^2-y^2_2 &\hdots & 0 \\
        y_1y_2 & \ddots & & 0\\
        \vdots &  & -y^2_{n-1}+y^2_{n-2} & -x_{n}y_{n-1} \\
        0 & \hdots & x_{n-1}y_{n-1} & y^2_{n-1}

    \end{bmatrix}
\]
\\

Therefore, we get \\

\[[H,[H,N]] = 
\begin{bmatrix}
    2y_1^2 & y_1(x_2-x_1) & 0 \\
    y_1(x_2-x_1) & -2(y_1^2-y^2_2) & 0\\
    0 & \ddots & 0\\
    \vdots & & y_{n-1}(x_{n}-x_{n-1}) \\
    0 & \hdots & -2y^2_{n-1}
\end{bmatrix}
\]

This is equivalent to the result we get from combining the matrix form of $H$ with the values for $\dot{x}$ and $\dot{y}$
\[\dot{H} = \begin{bmatrix}
    \dot{x}_{1} & \dot{y}_{1} & 0  & \dots & 0 \\
    \dot{y}_{1} & \dot{x}_{2} & \dot{y}_{2} & \dots & 0 \\
     & & \ddots & \\
     & & \dot{y}_{n-2} & \dot{x}_{n-1} & \dot{y}_{n-1}\\
    0 & \hdots & & \dot{y}_{n-1} & \dot{x}_{n}
    
\end{bmatrix}\]

In total, we conclude that the double bracket notation of the Toda lattice used in other papers is analogous to the system of ODEs. We use the equivalent system of ODEs to construct the analog sorter.




%\documentclass{article}
%\usepackage[utf8]{inputenc}
%\usepackage{amsmath}

%\title{Toda Lattice, Role of $N$ and $y_k$}
%\author{Lusa Zhan}
%\date{November 2016}

%\begin{document}

%\maketitle

\subsubsection{Role of $N$}
The choice of $N = \text{diag} (n, n-1, \dots, 1)$ for analog sorting is clarified by Theorem 1.5 in~\cite{helmke}. To summarize, the theorem states the following: 

For each $N$,  $\dot{H}(t) = [H,[H,N]]$ converges to an equilibrium $H_\infty$ as $t\rightarrow \infty$. If $N=\text{diag} (\mu_1 , \dots , \mu_n)$ where $\mu_1 > \dots > \mu_n$, then the Hessian of $f_N(H) = \frac{1}{2}\|N-H\|^2$ is nonsingular and negative definite. 

The linearization of the double bracket flow at an equilibrium point $H_\infty = \text{diag}(\lambda_{\pi(1)},\dots,\lambda_{\pi(n)})$ is 
\[\dot \xi_{ij} = -(\lambda_{\pi(i)}-\lambda_{\pi(j)})(\mu_i-\mu_j)\xi_{ij}\]
where $\xi = [H_\infty, N]$. Since the Hessian is negative definite, $\xi$ must be at its maximum. If $N = \text{diag} (n, n-1, \dots, 1)$, then $\xi$ will only reach its maximum if the diagonal entries of $H_\infty$ are sorted in descending order as well.

Thus the diagonal matrix $N$ determines the order of the diagonal entries in $H_\infty$. Letting $N = \text{diag}(n, ,n-1, \dots, 1)$, we will result in $x_1 \geq x_2 \geq \cdots \geq x_n$ in $H_\infty$.

\subsubsection{Role of off-diagonals $y_k$}
The choice of initial values for the off-diagonal $y_k$ values affects the results of analog sorting as well. 

First of all, as mentioned previously, $y_0=y_n=0$.
To analyze the initial values of $y_k$, consider the relation to $\dot x_k$
\begin{align*}
    \dot{x_k} &=  2y^2_k-2y^2_{k-1} \\
    \dot{y_k} &= y_k(x_{k+1}-x_k) 
\end{align*}

If $y_k=0$, then $\dot x_k=0$, meaning $x_k(t) = x_k(0)$ for all $t$. This means that the values along the diagonal of $H$ will remain constant and never get sorted. So, $y_k$ should be non-zero.

In addition, the $y_k$ values should be small (but still positive). Recall that the eigenvalues will be the values along the diagonal of $H$ as $t\rightarrow \infty$. Consider the determinant $f_n = \det (H-\lambda I)$, where $n$ is the size of the matrix. Since $H$ is a tridiagonal matrix, this determinant can be formulated using a recurrence relation on the size of the matrix $n$.

\begin{align*}
f_n &= (x_n-\lambda)f_{n-1}-y_{n-1}^2f_{n-2} \\
f_1 &= x_1-\lambda
\end{align*}

We can use this to calculate the first few $f_n$ and set them to $0$ (what we would do to calculate the eigenvalues):
\begin{align*}
f_1 &= x_1-\lambda = 0\\
f_2 &= (x_2-\lambda)(x_1-\lambda)-y_1^2 = 0\\
f_3 &= (x_3-\lambda)(x_2-\lambda)(x_1-\lambda)-(x_3-\lambda)y_1^2-(x_1-\lambda)y^2_2 = 0
\end{align*}

Intuitively, the eigenvalues will be closest to $x_k$ if the $y_k$ values are smaller. When setting up the analog sorter, we want the off-diagonals to be nonzero, but small enough to not significantly change the values to be sorted. 



%\section{Related work}
%Below is a list of articles that are related to this topic: 
%\begin{itemize}

%\item{
%    Brockett, R.w. ``Dynamical Systems That Sort Lists, Diagonalize Matrices, and Solve Linear Programming Problems.'' Linear Algebra and Its Applications 146 (1991): 79-91. }
%\item{
%    Helmke, Uwe, and John B. Moore. ``Double Bracket Isospectral Flows.'' Optimization and Dynamical Systems. London: Springer-Verlag, 1994. 43-80. Print.}
%\end{itemize}


%\end{document}


% equivalently, the Toda flow is
% dX/dt = [X(t), pi_0(X(t))]
% =X(t) pi_0(X(t)) - pi_0(X(t)) X(t)

% where pi_0 = X^- - X^{-T}
% where X^- is the lower triangular part
