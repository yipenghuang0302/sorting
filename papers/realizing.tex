After reviewing the mathematics of analog sorting, we now discuss how the analog sorter is put in practice.

The analog sorter works as follows:
% We set up a vector x, consisting of real numbers in a jumbled order.
We construct matrix $H(t=0)$ with the real numbers we want to sort on the diagonal.
Naturally $H(t=0)$ has eigenvalues consisting of the same real numbers.
We set up a special ODE that involves the vector $x$, and another vector consisting of the natural numbers.
This vector of the natural numbers provides the ``discreteness'' for the algorithm.
Specifically, this ODE is the finite Toda lattice ODE, which preserves the eigenvalues of $H(t)$, while reordering the elements on the diagonal to the sorted sequence.
This solves this ODE on an analog computer.
The final steady state of the analog output would have the original elements of the vector $x$, but now in sorted order.
For example, the first integrator would have the lowest magnitude element of $x$.

% It's still unclear to me:
% 1. What's the role of the off-diagonal parts of the symmetric, tridiagonal H. Brockett tells us to put some small values there to kick off the algorithm.
% 2. What's the role of the matrix of the natural numbers N.
% The matrix N of the natural numbers happens to make [H,N] have the form B on page 37 of Bloch & Rojo.
% Now, we can say that matrix N is the way it is just to make the HN-NH have a special shape.

% But, on the other hand, N, being a matrix of integers, is the only participant in this system that has "discreteness." Otherwise the system is all real numbers evolving in continuous-time.

There are some variations to this idea.
One is we can tackle the finite Toda lattice system directly as a Hamiltonian system.
Another is we can reverse the roles of the sort keys and the indices.
This is described by Brockett briefly in page 802 of~\cite{brockett}.
% Way 1:
% H(0) is the tridiagonal, with the stuff we want to sort on the diagonal. Off diagonal, some small values. N is diagonal matrix of the natural numbers.
% Way 2:
% N is diagonal matrix of the stuff we want to sort.
% swapping two vars
% decreases the sparsity
% but this is important because this way you move the indexes, instead of keys